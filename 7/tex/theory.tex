\documentclass[11pt,fleqn]{article}
\usepackage[cm]{fullpage}
\usepackage{mathtools} %includes amsmath
\usepackage{amsfonts}
\usepackage{bm}
\usepackage{xfrac}
\usepackage{url}
%greek letters
\renewcommand{\a}{\alpha}    %alpha
\renewcommand{\b}{\beta}     %beta
\newcommand{\g}{\gamma}      %gamma
\newcommand{\G}{\Gamma}      %Gamma
\renewcommand{\d}{\delta}    %delta
\newcommand{\D}{\Delta}      %Delta
\newcommand{\e}{\varepsilon} %epsilon
\newcommand{\ev}{\epsilon}   %epsilon*
\newcommand{\z}{\zeta}       %zeta
\newcommand{\h}{\eta}        %eta
\renewcommand{\th}{\theta}   %theta
\newcommand{\Th}{\Theta}     %Theta
\newcommand{\io}{\iota}      %iota
\renewcommand{\k}{\kappa}    %kappa
\newcommand{\la}{\lambda}    %lambda
\newcommand{\La}{\Lambda}    %Lambda
\newcommand{\m}{\mu}         %mu
\newcommand{\n}{\nu}         %nu %xi %Xi %pi %Pi
\newcommand{\p}{\rho}        %rho
\newcommand{\si}{\sigma}     %sigma
\newcommand{\siv}{\varsigma} %sigma*
\newcommand{\Si}{\Sigma}     %Sigma
\renewcommand{\t}{\tau}      %tau
\newcommand{\up}{\upsilon}   %upsilon
\newcommand{\f}{\phi}        %phi
\newcommand{\F}{\Phi}        %Phi
\newcommand{\x}{\chi}        %chi
\newcommand{\y}{\psi}        %psi
\newcommand{\Y}{\Psi}        %Psi
\newcommand{\w}{\omega}      %omega
\newcommand{\W}{\Omega}      %Omega
%ornaments
\newcommand{\eth}{\ensuremath{^\text{th}}}
\newcommand{\rst}{\ensuremath{^\text{st}}}
\newcommand{\ond}{\ensuremath{^\text{nd}}}
\newcommand{\ord}[1]{\ensuremath{^{(#1)}}}
\newcommand{\dg}{\ensuremath{^\dagger}}
\newcommand{\bigo}{\ensuremath{\mathcal{O}}}
\newcommand{\tl}{\ensuremath{\tilde}}
\newcommand{\ol}[1]{\ensuremath{\overline{#1}}}
\newcommand{\ul}[1]{\ensuremath{\underline{#1}}}
\newcommand{\op}[1]{\ensuremath{\hat{#1}}}
\newcommand{\ot}{\ensuremath{\otimes}}
\newcommand{\wg}{\ensuremath{\wedge}}
%text
\newcommand{\tr}{\ensuremath{\hspace{1pt}\mathrm{tr}\hspace{1pt}}}
\newcommand{\Alt}{\ensuremath{\mathrm{Alt}}}
\newcommand{\sgn}{\ensuremath{\mathrm{sgn}}}
\newcommand{\occ}{\ensuremath{\mathrm{occ}}}
\newcommand{\vir}{\ensuremath{\mathrm{vir}}}
\newcommand{\spn}{\ensuremath{\mathrm{span}}}
\newcommand{\vac}{\ensuremath{\mathrm{vac}}}
\newcommand{\bs}{\ensuremath{\text{\textbackslash}}}
\newcommand{\im}{\ensuremath{\mathrm{im}\hspace{1pt}}}
\renewcommand{\sp}{\hspace{30pt}}
%dots
\newcommand{\ld}{\ensuremath{\ldots}}
\newcommand{\cd}{\ensuremath{\cdots}}
\newcommand{\vd}{\ensuremath{\vdots}}
\newcommand{\dd}{\ensuremath{\ddots}}
\newcommand{\etc}{\ensuremath{\mathinner{\mkern-1mu\cdotp\mkern-2mu\cdotp\mkern-2mu\cdotp\mkern-1mu}}}
%fonts
\newcommand{\bmit}[1]{{\bfseries\itshape\mathversion{bold}#1}}
\newcommand{\mc}[1]{\ensuremath{\mathcal{#1}}}
\newcommand{\mb}[1]{\ensuremath{\mathbb{#1}}}
\newcommand{\mf}[1]{\ensuremath{\mathfrak{#1}}}
\newcommand{\mr}[1]{\ensuremath{\mathrm{#1}}}
\newcommand{\bo}[1]{\ensuremath{\mathbf{#1}}}
%styles
\newcommand{\ts}{\textstyle}
\newcommand{\ds}{\displaystyle}
\newcommand{\phsub}{\ensuremath{_{\phantom{p}}}}
\newcommand{\phsup}{\ensuremath{^{\phantom{p}}}}
%fractions, derivatives, parentheses, brackets, etc.
\newcommand{\pr}[1]{\ensuremath{\left(#1\right)}}
\newcommand{\brk}[1]{\ensuremath{\left[#1\right]}}
\newcommand{\fr}[2]{\ensuremath{\dfrac{#1}{#2}}}
\newcommand{\pd}[2]{\ensuremath{\frac{\partial#1}{\partial#2}}}
\newcommand{\pt}{\ensuremath{\partial}}
\newcommand{\br}[1]{\ensuremath{\langle#1|}}
\newcommand{\kt}[1]{\ensuremath{|#1\rangle}}
\newcommand{\ip}[1]{\ensuremath{\langle#1\rangle}}
\newcommand{\NO}[1]{\ensuremath{{\bm{:}}#1{}{\bm{:}}}}
\newcommand{\floor}[1]{\ensuremath{\left\lfloor#1\right\rfloor}}
\newcommand{\ceil}[1]{\ensuremath{\left\lceil#1\right\rceil}}
\usepackage{stackengine}
\newcommand{\GNO}[1]{\setstackgap{S}{0.7pt}\ensuremath{\Shortstack{\textbf{.} \textbf{.} \textbf{.}}#1\Shortstack{\textbf{.} \textbf{.} \textbf{.}}}}
\newcommand{\cmtr}[2]{\ensuremath{[\cdot,#2]^{#1}}}
\newcommand{\cmtl}[2]{\ensuremath{[#2,\cdot]^{#1}}}
%structures
\newcommand{\eqn}[1]{(\ref{#1})}
\newcommand{\ma}[1]{\ensuremath{\begin{bmatrix}#1\end{bmatrix}}}
\newcommand{\ar}[1]{\ensuremath{\begin{matrix}#1\end{matrix}}}
\newcommand{\miniar}[1]{\ensuremath{\begin{smallmatrix}#1\end{smallmatrix}}}
%contractions
\usepackage{simplewick}
\usepackage[nomessages]{fp}
\newcommand{\ctr}[6][0]{\FPeval\height{0.6+#1*0.5}\ensuremath{\contraction[\height ex]{#2}{#3}{#4}{#5}}}
\newcommand{\ccr}[4]{\ctr[0.7]{#1}{#2}{#3}{#4}{}\ctr{#1}{#2}{#3}{#4}}
\usepackage{calc}
\makeatletter
\def\@hspace#1{\begingroup\setlength\dimen@{#1}\hskip\dimen@\endgroup}
\makeatother
\newcommand{\halfphantom}[1]{\hspace{\widthof{#1}*\real{0.5}}}
\newcommand{\fullctr}[1]{\ensuremath{\contraction[0.5ex]{}{\vphantom{#1}}{\hphantom{#1}}{}{}{}\contraction[0.5ex]{}{\vphantom{#1}}{\halfphantom{#1}}{}{}{}#1}}
%math sections
\usepackage{cleveref}
\usepackage{amsthm}
\usepackage{thmtools}
\declaretheoremstyle[spaceabove=10pt,spacebelow=10pt,bodyfont=\small]{mystyle}
\theoremstyle{mystyle}
\newtheorem{dfn}{Definition}[section]
\crefname{dfn}{definition}{definitions}
\Crefname{dfn}{Def}{Defs}
\newtheorem{thm}{Theorem}[section]
\crefname{thm}{theorem}{theorems}
\Crefname{thm}{Thm}{Thms}
\newtheorem{cor}{Corollary}[section]
\crefname{cor}{corollary}{corollaries}
\Crefname{cor}{Cor}{Cors}
\newtheorem{lem}{Lemma}[section]
\crefname{lem}{lemma}{lemmas}
\Crefname{lem}{Lem}{Lems}
\newtheorem{rmk}{Remark}[section]
\crefname{rmk}{remark}{remarks}
\Crefname{rmk}{Rmk}{Rmks}
\newtheorem{pro}{Proposition}[section]
\crefname{pro}{proposition}{propositions}
\Crefname{pro}{Prop}{Props}
\newtheorem{ntt}{Notation}[section]
\crefname{ntt}{notation}{notations}
\Crefname{ntt}{Notation}{Notations}

\usepackage{cancel}
\usepackage{scrextend}

\newcommand{\hole}{\circ}
\newcommand{\ptcl}{\bullet}
\newcommand{\circled}[1]{\raisebox{.5pt}{\textcircled{\raisebox{-.9pt}{#1}}}}


\title{Programming Project 7 Theory\\
\textit{Deriving CIS matrix elements in Kutzelnigg-Mukherjee tensor notation}}
\date{}
\author{}

\begin{document}
\maketitle
\vspace{-1cm}

\noindent
The electronic Hamiltonian can be expressed as follows.
\begin{align}
&&
  H_e
=
  E_{\text{HF}}
+
  H_c
&&
  E_{\text{HF}}
=
  \ip{\F|H_e|\F}
&&
  H_c
=
  f_p^q
  \tl{a}_q^p
+
  \tfrac{1}{4}
  \ol{g}_{pq}^{rs}
  \tl{a}_{rs}^{pq}
\end{align}
Assuming Brillouin's theorem is satisfied and $\ip{\F|H_e|\F_i^a}=f_i^a=0$, the CIS ground state eigenpair is simply the Hartree-Fock solution: the root is $E_{\text{HF}}$ and the wavefunction is $\F$.
Therefore, excitation energies from the ground state are eigenvalues of the matrix $\ip{\F_i^a|H_e-E_{\text{HF}}|\F_j^b}=\ip{\F_i^a|H_c|\F_j^b}$.
Applying Wick's theorem in $\F$-normal ordering gives
\begin{align}
&&\nonumber
  \ip{\F_i^a|H_c|\F_j^b}
=&\
  f_p^q
  \ip{\F|\tl{a}_a^i\tl{a}_q^p\tl{a}_j^b|\F}
+
  \tfrac{1}{4}
  \ol{g}_{pq}^{rs}
  \ip{\F|\tl{a}_a^i\tl{a}_{rs}^{pq}\tl{a}_j^b|\F}
\\&&=&\
\label{cis-matrix-elements}
  f_p^q
  (\tl{a}_a^i\tl{a}_q^p\tl{a}_j^b)_{\text{f.c.}}
+
  \tfrac{1}{4}
  \ol{g}_{pq}^{rs}
  (\tl{a}_a^i\tl{a}_{rs}^{pq}\tl{a}_j^b)_{\text{f.c.}}
\end{align}
where the subscript f.c. denotes the sum over fully contracted terms in these Wick expansions.
\begin{align}
\label{one-electron-coupling-matrix}
&
  (\tl{a}_a^i\tl{a}_q^p\tl{a}_j^b)_{\text{f.c.}}
=
  \GNO{a_{a^\hole}^{i^\ptcl}a_{q^{\hole\hole}}^{p^\hole}a_{j^\ptcl}^{b^{\hole\hole}}}
+
  \GNO{a_{a^\hole}^{i^{\ptcl}}a_{q^{\ptcl}}^{p^{\ptcl\ptcl}}a_{j^{\ptcl\ptcl}}^{b^{\hole}}}
=
  \g^i_j\h_a^p\h_q^b
-
  \g_q^i\g_j^p\h_a^b
\\
\label{two-electron-coupling-matrix}
&
  (\tl{a}_a^i\tl{a}_{rs}^{pq}\tl{a}_j^b)_{\text{f.c.}}
=
  \GNO{
    \tl{a}_{a^\hole}^{i^\ptcl}
    \tl{a}_{r^{\ptcl}s^{\hole\hole}}^{p^{\hole}q^{\ptcl\ptcl}}
    \tl{a}_{j^{\ptcl\ptcl}}^{b^{\hole\hole}}
  }
+
  \GNO{
    \tl{a}_{a^\hole}^{i^\ptcl}
    \tl{a}_{r^{\hole\hole}s^{\ptcl}}^{p^{\hole}q^{\ptcl\ptcl}}
    \tl{a}_{j^{\ptcl\ptcl}}^{b^{\hole\hole}}
  }
+
  \GNO{
    \tl{a}_{a^\hole}^{i^\ptcl}
    \tl{a}_{r^{\ptcl}s^{\hole\hole}}^{p^{\ptcl\ptcl}q^{\hole}}
    \tl{a}_{j^{\ptcl\ptcl}}^{b^{\hole\hole}}
  }
+
  \GNO{
    \tl{a}_{a^\hole}^{i^\ptcl}
    \tl{a}_{r^{\hole\hole}s^{\ptcl}}^{p^{\ptcl\ptcl}q^{\hole}}
    \tl{a}_{j^{\ptcl\ptcl}}^{b^{\hole\hole}}
  }
=
  \op{P}^{(p/q)}_{(r/s)}
  \g^i_r\h^p_a\g^q_j\h^b_s
\end{align}
Plugging equations \ref{one-electron-coupling-matrix} and \ref{two-electron-coupling-matrix} into equation \ref{cis-matrix-elements} gives the final working equation for the CIS matrix elements.
\begin{align}
\label{cis-equations}
&&
  \ip{\F_i^a|H_c|\F_j^b}
=
  f_a^b\g_j^i
-
  f_j^i\h_a^b
+
  \ol{g}_{aj}^{ib}
=
  f_a^b\d_j^i
-
  f_j^i\d_a^b
+
  \ol{g}_{aj}^{ib}
\end{align}
For a canonical Hartree-Fock reference, the Fock matrix is diagonal: $f_a^b=\ev_a\d_a^b$ and $f_j^i=\ev_j\d_j^i$.



\end{document}