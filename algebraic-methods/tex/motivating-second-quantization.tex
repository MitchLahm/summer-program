\documentclass[11pt,fleqn]{article}
\usepackage[cm]{fullpage}
\usepackage{mathtools} %includes amsmath
\usepackage{amsfonts}
\usepackage{bm}
\usepackage{url}
%greek letters
\renewcommand{\a}{\alpha}    %alpha
\renewcommand{\b}{\beta}     %beta
\newcommand{\g}{\gamma}      %gamma
\newcommand{\G}{\Gamma}      %Gamma
\renewcommand{\d}{\delta}    %delta
\newcommand{\D}{\Delta}      %Delta
\newcommand{\e}{\varepsilon} %epsilon
\newcommand{\ev}{\epsilon}   %epsilon*
\newcommand{\z}{\zeta}       %zeta
\newcommand{\h}{\eta}        %eta
\renewcommand{\th}{\theta}   %theta
\newcommand{\Th}{\Theta}     %Theta
\newcommand{\io}{\iota}      %iota
\renewcommand{\k}{\kappa}    %kappa
\newcommand{\la}{\lambda}    %lambda
\newcommand{\La}{\Lambda}    %Lambda
\newcommand{\m}{\mu}         %mu
\newcommand{\n}{\nu}         %nu %xi %Xi %pi %Pi
\newcommand{\p}{\rho}        %rho
\newcommand{\si}{\sigma}     %sigma
\newcommand{\siv}{\varsigma} %sigma*
\newcommand{\Si}{\Sigma}     %Sigma
\renewcommand{\t}{\tau}      %tau
\newcommand{\up}{\upsilon}   %upsilon
\newcommand{\f}{\phi}        %phi
\newcommand{\F}{\Phi}        %Phi
\newcommand{\x}{\chi}        %chi
\newcommand{\y}{\psi}        %psi
\newcommand{\Y}{\Psi}        %Psi
\newcommand{\w}{\omega}      %omega
\newcommand{\W}{\Omega}      %Omega
%ornaments
\newcommand{\eth}{\ensuremath{^\text{th}}}
\newcommand{\rst}{\ensuremath{^\text{st}}}
\newcommand{\ond}{\ensuremath{^\text{nd}}}
\newcommand{\ord}[1]{\ensuremath{^{(#1)}}}
\newcommand{\dg}{\ensuremath{^\dagger}}
\newcommand{\bigo}{\ensuremath{\mathcal{O}}}
\newcommand{\tl}{\ensuremath{\tilde}}
\newcommand{\ol}[1]{\ensuremath{\overline{#1}}}
\newcommand{\ul}[1]{\ensuremath{\underline{#1}}}
\newcommand{\op}[1]{\ensuremath{\hat{#1}}}
\newcommand{\ot}{\ensuremath{\otimes}}
\newcommand{\wg}{\ensuremath{\wedge}}
%text
\newcommand{\tr}{\ensuremath{\hspace{1pt}\mathrm{tr}\hspace{1pt}}}
\newcommand{\Alt}{\ensuremath{\mathrm{Alt}}}
\newcommand{\sgn}{\ensuremath{\mathrm{sgn}}}
\newcommand{\occ}{\ensuremath{\mathrm{occ}}}
\newcommand{\vir}{\ensuremath{\mathrm{vir}}}
\newcommand{\spn}{\ensuremath{\mathrm{span}}}
\newcommand{\vac}{\ensuremath{\mathrm{vac}}}
\newcommand{\bs}{\ensuremath{\text{\textbackslash}}}
\newcommand{\im}{\ensuremath{\mathrm{im}\hspace{1pt}}}
\renewcommand{\sp}{\hspace{30pt}}
%dots
\newcommand{\ld}{\ensuremath{\ldots}}
\newcommand{\cd}{\ensuremath{\cdots}}
\newcommand{\vd}{\ensuremath{\vdots}}
\newcommand{\dd}{\ensuremath{\ddots}}
\newcommand{\etc}{\ensuremath{\mathinner{\mkern-1mu\cdotp\mkern-2mu\cdotp\mkern-2mu\cdotp\mkern-1mu}}}
%fonts
\newcommand{\bmit}[1]{{\bfseries\itshape\mathversion{bold}#1}}
\newcommand{\mc}[1]{\ensuremath{\mathcal{#1}}}
\newcommand{\mb}[1]{\ensuremath{\mathbb{#1}}}
\newcommand{\mf}[1]{\ensuremath{\mathfrak{#1}}}
\newcommand{\mr}[1]{\ensuremath{\mathrm{#1}}}
\newcommand{\bo}[1]{\ensuremath{\mathbf{#1}}}
%styles
\newcommand{\ts}{\textstyle}
\newcommand{\ds}{\displaystyle}
\newcommand{\phsub}{\ensuremath{_{\phantom{p}}}}
\newcommand{\phsup}{\ensuremath{^{\phantom{p}}}}
%fractions, derivatives, parentheses, brackets, etc.
\newcommand{\pr}[1]{\ensuremath{\left(#1\right)}}
\newcommand{\brk}[1]{\ensuremath{\left[#1\right]}}
\newcommand{\fr}[2]{\ensuremath{\dfrac{#1}{#2}}}
\newcommand{\pd}[2]{\ensuremath{\frac{\partial#1}{\partial#2}}}
\newcommand{\pt}{\ensuremath{\partial}}
\newcommand{\br}[1]{\ensuremath{\langle#1|}}
\newcommand{\kt}[1]{\ensuremath{|#1\rangle}}
\newcommand{\ip}[1]{\ensuremath{\langle#1\rangle}}
\newcommand{\NO}[1]{\ensuremath{{\bm{:}}#1{}{\bm{:}}}}
\usepackage{stackengine}
\newcommand{\GNO}[1]{\setstackgap{S}{0.7pt}\ensuremath{\Shortstack{\textbf{.} \textbf{.} \textbf{.}}#1\Shortstack{\textbf{.} \textbf{.} \textbf{.}}}}
\newcommand{\cmtr}[2]{\ensuremath{[\cdot,#2]^{#1}}}
\newcommand{\cmtl}[2]{\ensuremath{[#2,\cdot]^{#1}}}
%structures
\newcommand{\eqn}[1]{(\ref{#1})}
\newcommand{\ma}[1]{\ensuremath{\begin{bmatrix}#1\end{bmatrix}}}
\newcommand{\ar}[1]{\ensuremath{\begin{matrix}#1\end{matrix}}}
\newcommand{\miniar}[1]{\ensuremath{\begin{smallmatrix}#1\end{smallmatrix}}}
%contractions
\usepackage{simplewick}
\usepackage[nomessages]{fp}
\newcommand{\ctr}[6][0]{\FPeval\height{1.0+#1*0.5}\ensuremath{\contraction[\height ex]{#2}{#3}{#4}{#5}}}
%math sections
\usepackage{cleveref}
\usepackage{amsthm}
\usepackage{thmtools}
\declaretheoremstyle[spaceabove=10pt,spacebelow=10pt,bodyfont=\small]{mystyle}
\theoremstyle{mystyle}
\newtheorem{dfn}{Definition}[section]
\crefname{dfn}{definition}{definitions}
\Crefname{dfn}{Def}{Defs}
\newtheorem{thm}{Theorem}[section]
\crefname{thm}{theorem}{theorems}
\Crefname{thm}{Thm}{Thms}
\newtheorem{cor}{Corollary}[section]
\crefname{cor}{corollary}{corollaries}
\Crefname{cor}{Cor}{Cors}
\newtheorem{lem}{Lemma}[section]
\crefname{lem}{lemma}{lemmas}
\Crefname{lem}{Lem}{Lems}
\newtheorem{rmk}{Remark}[section]
\crefname{rmk}{remark}{remarks}
\Crefname{rmk}{Rmk}{Rmks}
\newtheorem{pro}{Proposition}[section]
\crefname{pro}{proposition}{propositions}
\Crefname{pro}{Prop}{Props}
\newtheorem{ntt}{Notation}[section]
\crefname{ntt}{notation}{notations}
\Crefname{ntt}{Notation}{Notations}
\usepackage{calc}
\newcommand{\logbox}{\ensuremath{\text{\makebox[\widthof{exp}][c]{ln}}}}
\newcommand{\expbox}{\ensuremath{\text{\makebox[\widthof{exp}][c]{exp}}}}


\usepackage{cancel}

%%%DOCUMENT%%%
\begin{document}

\section*{Motivating second quantization}

{\small Note: In what follows, the terminology ``$n$-electron function'' will refer to an \underline{antisymmetric} function of $n$ space-spin coordinates $(i)\equiv(\bo{r}_i,s_i)$.  We implicitly take every integral to be a definite integral over all values of $(\bo{r}_i,s_i)$.}\\


\noindent
\bmit{Annihilation operators.}
Let $\{\y_p\}$ be a complete basis of spin-orbitals and let $\F_{(p_1\cd p_n)}$ denote the $n$-electron Slater determinant formed from $\y_{p_1},\ld,\y_{p_n}$.
Then it is possible to define an operator $\op{a}_{p_1}$ which deletes $\y_{p_1}$ from $\F_{(p_1\cd p_n)}$ to produce $\F_{(p_2\cd p_n)}$, the $(n-1)$-electron Slater determinant formed from $\y_{p_2},\ld,\y_{p_n}$.
We can define such an ``annihilator'' $\op{a}_p$ of spin-orbital $\y_p$ explicitly as
\begin{align}
\label{a-func}
  (\op{a}_p\Y)(2,\ld,m)
\equiv
  \sqrt{m}
  \int d(1)\
  \y_p^*(1)\Y(1,2,\ld,m)
\end{align}
which takes an $m$-electron function $\Y$ into the $(m-1)$-electron function $\op{a}_p\Y$.
Applied to $\F_{(p_1\cd p_n)}$, this gives
\begin{align}
\label{a-det}
  \op{a}_p\F_{(p_1\cd p_n)}
=
  \left\{\ar{
    (-)^{k-1}\F_{(p_1\cd \cancel{p_k}\cd p_n)} & \text{if $p=p_k\in (p_1\cd p_n)$}\\[10pt]
    0 & \text{otherwise.}
  }\right.
\end{align}
As an exercise, it is worth proving to yourself that the definition in equation \ref{a-func} satisfies equation \ref{a-det}.\\

\noindent
\bmit{Annihilation operators anticommute.}
Consider the action of two annihilators $\op{a}_p$ and $\op{a}_q$ on an $n$-electron function $\Y$.
If $\op{a}_p$ is applied first, we obtain
\begin{align*}
  (\op{a}_q\op{a}_p\Y)(3,\ld,n)
=
  \sqrt{n(n-1)}
  \int d(1)d(2)\
  \y_p^*(1)\y_q^*(2)\Y(1,2,3,\ld,n)
\end{align*}
whereas if $\op{a}_q$ is applied first, we obtain
\begin{align*}
  (\op{a}_p\op{a}_q\Y)(3,\ld,n)
=&\
  \sqrt{n(n-1)}
  \int d(1)d(2)\
  \y_q^*(1)\y_p^*(2)\Y(1,2,3,\ld,n)\ .
\end{align*}
This last integral can be rewritten as
\begin{align*}
  \int d(1)d(2)\
  \y_p^*(1)\y_q^*(2)\Y(2,1,3,\ld,n)
=
-
  \int d(1)d(2)\
  \y_p^*(1)\y_q^*(2)\Y(1,2,3,\ld,n)
\end{align*}
by swapping dummy variables of integration ($1\leftrightarrow2$) and using the antisymmetry of $\Y$.
This shows that $\op{a}_p\op{a}_q\Y=-\op{a}_q\op{a}_p\Y$ and, restricting ourselves to the space of $m$-electron ($m=0,1,2,\ld,\infty$) functions, we have the identity
\begin{align}
&&
  [\op{a}_p,\op{a}_q]_+
\equiv
  \op{a}_p\op{a}_q + \op{a}_q\op{a}_p
=
  0
\end{align}
i.e. the annihilation operators anticommute.
Note that restriction to antisymmetric functions is key -- were we to restrict ourselves to \textit{symmetric} functions, equation \ref{a-func} would define operators which instead satisfy $\op{a}_p\op{a}_q-\op{a}_q\op{a}_p=0$.\\


\noindent
\bmit{$\Y$ decomposition.}
Using these newfangled operators, any $n$-electron function can be decomposed as
\begin{align}
\label{decomp}
  \Y(1,\cd,n)
=
  \fr{1}{\sqrt{n}}
  \sum_p^\infty
  \y_p(1)\ (\op{a}_p\Y)(2,\ld,n)
=
  \fr{1}{\sqrt{n(n-1)}}
  \sum_{pq}^\infty
  \y_p(1)\y_q(2)\ (\op{a}_q\op{a}_p\Y)(3,\ld,n)
\end{align}
by resolution of the identity along one or two sets of electron coordinates  (recall the definition in equation \ref{a-func}).\footnote{
In principle, this could be carried on to eliminate all of the coordinates from $\Y$.
The scalars $(\op{a}_{p_n}\cd\op{a}_{p_1}\Y)$ in this expansion would then be equal to $\sqrt{n!}\ c_{p_1\cd p_n}$, where $c_{p_1\cd p_n}$ is the expansion coefficient of $\Y$ in the basis of spin-orbital products.}

\noindent
\bmit{Expressing $\op{H}_e$ in terms of annihilation operators.}
The electronic Hamiltonian\footnote{Here I'm defining the electronic Hamiltonian as $\op{H}_e=\op{H}-(\op{V}_\text{Nu}+\op{T}_\text{Nu})$} of an $n$-electron system can be written as
\begin{align*}
  \op{H}_e
=
  \sum_i^n\op{h}(i)
+
  \sum_{i<j}^n\op{g}(i,j)
\sp
  \op{h}(i)
=
-
  \fr{1}{2}\nabla_i^2
+
  \sum_A \fr{Z_A}{|\bo{r}_i-\bo{R}_A|}
\sp
  \op{g}(i,j)
=
  \fr{1}{|\bo{r}_i-\bo{r}_j|}
\end{align*}
and its matrix elements with respect to arbitrary $n$-electron functions $\Y$ and $\Y'$ are given by
\begin{align*}
  \ip{\Y|\op{H}_e|\Y'}
=&\
  \sum_i^n\ip{\Y|\op{h}(i)|\Y'}
+
  \sum_{i<j}^n\ip{\Y|\op{g}(i,j)|\Y'}
\\=&\
  n\ip{\Y|\op{h}(1)|\Y'}
+
  \fr{n(n-1)}{2}\ip{\Y|\op{g}(1,2)|\Y'}
\end{align*}
where the second equality can be shown by exchanging dummy variables ($1\leftrightarrow i$ and $2\leftrightarrow j$) in each integral and noting that $\Y$ and $\Y'$ are antisymmetric.
Then, using the two decompositions shown in equation \ref{decomp}, we can write this as
\begin{align*}
  \ip{\Y|\op{H}_e|\Y'}
=
  \sum_{pq}^\infty h_{pq} \ip{\op{a}_p\Y|\op{a}_q\Y'}
+
  \fr{1}{2}
  \sum_{pqrs}^\infty \ip{pq|rs} \ip{\op{a}_q\op{a}_p\Y|\op{a}_s\op{a}_r\Y'}
\sp
  h_{pq}
=&\
  \ip{\y_p(1)|\op{h}(1)|\y_q(1)}
\\
  \ip{pq|rs}
=&\
  \ip{\y_p(1)\y_q(2)|\op{g}(1,2)|\y_r(1)\y_s(2)}
\end{align*}
where $h_{pq}$ and $\ip{pq|rs}$ are our one- and two-electron integrals.
Restricting ourselves to the space of $m$-electron ($m=0,1,2,\ld,\infty$) functions, we have the identity
\begin{align}
&&
  \op{H}_e
=
  \sum_{pq}^\infty
  h_{pq}
  \op{a}_p\dg \op{a}_q
+
  \sum_{pqrs}^\infty
  \ip{pq|rs}
  \op{a}_p\dg \op{a}_q\dg \op{a}_s\op{a}_r
\end{align}
noting that $\ip{\op{a}_p\Y|\op{a}_q\Y'}=\ip{\Y|\op{a}_p\dg\op{a}_q\Y'}$ and $\ip{\op{a}_q\op{a}_p\Y|\op{a}_s\op{a}_r\Y'}=\ip{\Y|\op{a}_p\dg \op{a}_q\dg \op{a}_s\op{a}_r\Y'}$ by the definition of adjoint.\\



\end{document}