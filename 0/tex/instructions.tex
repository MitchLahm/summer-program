\documentclass[fleqn]{article}
\usepackage[cm]{fullpage}
\usepackage{mathtools} %includes amsmath
\usepackage{amsfonts}
\usepackage{bm}
\usepackage{url}

\usepackage{listings}
\lstset{basicstyle=\ttfamily\small}
\lstset{literate={~} {$\sim$}{1}}
\lstset{showstringspaces=false}
\lstset{language=Python}
\newcommand{\linp}[1]{\lstinputlisting{#1}}
\newcommand{\linl}[1]{\lstinline{#1}{}}
\newcommand{\ul}[1]{\underline{#1}}

\title{Programming Project 0: Geometry\\
\textit{Object-oriented programming basics}}
\author{}
\date{}

\begin{document}

\maketitle
\vspace{-1cm}
\noindent
Write a class to store the geometry, masses, and nuclear charges of a molecule.
Afterwards, write a small program to test your class.
Make sure each class method is called at least once.

\subsection*{Extra Files}
\begin{center}
\begin{tabular}{p{0.15\textwidth}@{}p{0.85\textwidth}}
  \ul{file name} & \ul{description} \\
  \linl{molecule.xyz}
  & sample \linl{.xyz} file for testing out your program \\ 
  \linl{masses.py}
  & provides functions \linl{get_mass} and \linl{get_charge} which return the mass and charge of an atom, e.g.:
  \begin{lstlisting}[language=python]
>>> import masses
>>> masses.get_mass("O")
15.99491461956
>>> masses.get_charge("O")
8
\end{lstlisting}
\end{tabular}
\end{center}

\subsection*{Class description}

\paragraph{Member variables.}
\begin{center}
\begin{tabular}{p{0.1\textwidth}@{}p{0.15\textwidth}@{}p{0.75\textwidth}}
  \ul{name} & \ul{type} & \ul{description} \\
  \linl{units}   & \linl{str}
  & either \linl{"Angstrom"} or \linl{"Bohr"}, specifying the distance units used for spatial coordinates\\
  \linl{natom}   & \linl{int}
  & the number of atoms\\
  \linl{labels}  & \linl{list} of \linl{str}s
  & a list of uppercase atomic symbols, following the order of the \linl{.xyz} file\\
  \linl{masses}  & \linl{list} of \linl{float}s
  & a list of atomic masses, following the order of the \linl{.xyz} file\\
  \linl{charges} & \linl{list} of \linl{int}s
  & a list of atomic charges, following the order of the \linl{.xyz} file\\
  \linl{geom}    & \linl{numpy.array}
  & an $\text{\linl{natom}}\times3$ matrix containing the Cartesian coordinates of each atom, following the order of the \linl{.xyz} file
\end{tabular}
\end{center}

\paragraph{Methods.}
\begin{center}
\begin{tabular}{p{0.2\textwidth}@{}p{0.8\textwidth}}
  \ul{method} & \ul{description}\\
  Constructor (\linl{__init__})
  & takes \linl{str} contents of an \linl{.xyz} file as input; initializes all member variables and fills them with their correct values\\
  \linl{to_bohr}
  & converts the distance units to Bohr, changing member variables \linl{units} and \linl{geom} if necessary \\
  \linl{to_angstrom}
  & converts the distance units to Angstroms, changing member variables \linl{units} and \linl{geom} if necessary \\
  \linl{xyz_string}
  & return \linl{str} representing the contents of the \linl{Molecule} object in \linl{.xyz} format\\
  \linl{copy}
  & returns \linl{Molecule} object, which is a fresh copy of \linl{self}
\end{tabular}
\end{center}


\end{document}